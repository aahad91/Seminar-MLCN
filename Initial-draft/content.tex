%%%%%%%%%%%%%%%%%%%%%%%%%%%%%%%%%%%%%%
% START ADDING TEXT HERE
%
% Feel free to use \include commands to structure text in smaller
% pieces
%
%%%%%%%%%%%%%%%%%%%%%%%%%%%%%%%%%%%%%%


% Abstract gives a brief summary of the main points of a paper:
\begin{abstract}
\textbf{Abstract - }
\end{abstract}

% the actual content, usually separated over a number of sections
% each section is assigned a label, in order to be able to put a
% crossreference to it

\section{Introduction}
\label{sec:introduction}
\begin{itemize}
  \item Starting with the IoT devices
  \item Management of IoT devices, transitioning from Cloud to Fog Computing
  \item some use-cases of IoT using fog computing
  \item Fog computing Infrastructure
  \item VMs vs Container-based solutions (VM Cloud infra vs Kubernetes)
  \item Kubernetes brief overview
  \item Kubernetes in-terms of Fog Computing (network consideration and issues)
  \item last paragraph about proposed solution for network related issues, and strucutre of seminar paper
\end{itemize}
In recent years with the evolution of technology, Internet of Things (IoT) devices are increasing day by day. According to Ericsson mobility report\cite{}, there will be 17\% (approx. 22.3 billion) increase in IoT devices by 2024. Functionally IoT is defined as "The Internet of Things allows people and things to be connected Anytime, Anyplace, with Anything and Anyone" \cite{European commission 2008}. IoT devices have served mankind in ways such as from smart houses to smart cities, smart transportation systems and many medical applications. These IoT applications enables many devices connected to network and generates alot of heterogeneous data also known as BigData which requires special data processing models and Infrastructure support.
But with the increasing number of devices, orchestration, communication between these devices and data generation are one of the main problems that need to be addressed. 

\section{Backgroud}
\label{sec:backgroud}
\begin{itemize}
  \item Kubernetes Internal Archietecture and Main Components
  \item Kubernetes works as an Orchestrator
  \item Kubernetes resource provisioning
  \item Concluding the section with pitfals of default scheduler of Kubernetes
\end{itemize}

\subsection{Kubernetes Main Components}
\label{sec:k8s_main_comp}
\begin{itemize}
  \item Write about the Archietecture of Kubernetes with diagram
  \item Write about the building blocks of Kubernetes and their working
\end{itemize}

\subsection{Kubernetes as Orchestrator}
\label{sec:k8s_orchestrator}
\begin{itemize}
  \item Orchestrator main functions
  \item Comparison of available Orchestrator (OpenStack vs Kubernetes)
  \item Workflow of Kubernetes as an Orchestrator (steps)
\end{itemize}

\subsection{Kubernetes Resource Provisioning}
\label{sec:k8s_scheduler}
\begin{itemize}
  \item write about the default Kubernetes scheduler
  \item its main Components
  \item workflow of default scheduler
\end{itemize}

\section{Kubernetes Network-based Resource Provisioning}
\label{sec: k8s_ns}
\begin{itemize}
  \item write about why we need network-based resource provisioning
  \item main factors consideration (e.g bandwidth and latency)
  \item workflow of network-based scheduler
\end{itemize}

\section{Performance Evaluation}
\label{sec:Performance_eval}
\begin{itemize}
  \item Write about the considered use-case of Fog Computing for Evaluation
\end{itemize}
\subsection{Expermentation Setup}
\label{sec:setup}
\begin{itemize}
  \item setup of Kubernetes base on the mentioned use-case of Fog Computing with diagram
\end{itemize}

\subsection{Analysis of Kubernetes Default and Network-based Resource Provisioning}
\label{sec:analysis}
\begin{itemize}
  \item write about the Performance difference between default Kubernetes scheduler and network based scheduler with supporting result tables and graphs
\end{itemize}

\section{Comparison of Network-based Resource Provisioning Solutions}
\label{sec:related_work}
\begin{itemize}
  \item Compare different solutions based on the following criteria:
\end{itemize}

\subsection{Orchestrator}
\label{sec:infra}
\begin{itemize}
  \item write about the differences between Kubernetes(main-paper)\cite{Santos2019} and other available cloud solutions such as Fogernetes\cite{Wobker2018} and \cite{Reale}.
\end{itemize}

\subsection{Resource Provisioning Techniques}
\begin{itemize}
  \item difference between different resource scheduling techniques such  as \cite{Bittencourt2017}, \cite{Haja2019} etc.
\end{itemize}

\section{Conclusion}
\label{sec:concl}

\section{Further Research Topics}
\label{sec:research}
\begin{itemize}
  \item after writing the seminar, if there is any improvement that can be done, will be added in this section.
\end{itemize}
